\documentclass[UTF8]{ctexrep}

\title{人工智能}
\author{}
\date{}
\usepackage[hidelinks]{hyperref}
\usepackage{listings}
\usepackage{ctex}
\usepackage{mathtools}
\usepackage{graphics}
\usepackage{extarrows}
\usepackage{amsmath}
\usepackage{listings}

\begin{document}

\maketitle
\tableofcontents
\newpage

\chapter{优化算法}
\section{mini-batch梯度下降法}
使用好的优化算法可以提高效率

把所以的$m$个样本都放在一个矩阵里面
$$X=[x^{(1)},x^{(2)},x^{(3)},\dots,x^{(m)}]$$
$$Y=[y^{(1)},y^{(2)},y^{(3)},\dots,y^{(m)}]$$
以上$X$的维数是$(n_x,m)$,$Y$的维数是$(1,m)$

这样的向量化可以很快的处理样本,但是如果$m$很大的话,处理速度仍然会很慢
(这里的$m$是样本的数量,每一个$x_i$的维度是$x_n$)

我们可以先让梯度下降分处理一部分的训练集,这样可以加快训练速度

进一步,可以把一个训练集分割成多个小的训练集,这些小的子集被称为Mini-batch 

这些小的子集,依顺序命名为$X^{\{t\}}$和$Y^{\{t\}}$

\begin{enumerate}
    \item mini-batch size=m : 这个便是之前的batch梯度下降法\\
    \item mini-batch size=1 : 这种算法叫做随机梯度下降法
\end{enumerate}
mini-batch梯度下降法 的size 大小在上述两种情况之间

如果样本的数量小于2000个,可以直接使用 batch梯度下降 算法

mini-batch size通常取$[64,512]$之间,而且取的是$2$的倍数,这样代码运行速度会快一点

$X^{\{t\}}$和$Y^{\{t\}}$的大小通常要符合CPU或者GPU的内存(这里我的理解是,CPU是内存,GPU是显存)


\section{指数加权移动平均值}
例如:

\begin{align*}
    \theta_1&=40^\circ F\\
    \theta_2&=49^\circ F\\
    \theta_3&=45^\circ F\\
    &\ \ \vdots\\
    \theta_{180}&=60^\circ F\\
    \theta_{181}&=56^\circ F\\
    &\ \ \vdots\\
\end{align*}

加权后:

    \begin{align*}
        v_0&=0\\
        v_1&=0.9v_0+0.1\theta_1\\
        &\ \ \vdots\\
        v_t&=0.9v_{t-1}+0.1\theta_t
    \end{align*}

即:
$$v_t=\beta v_{t-1}+(1-\beta)\theta_t$$

可以把$v_t$认为是$t$之前$\frac1{1-\beta}$个数据的平均值

当$\beta$的值越大的时候,点$V_t$所构成的曲线就越平滑,但是曲线会整体向右移动

$$v_\theta = 0$$
$$v_\theta := \beta v+(1-\beta)\theta_t$$

\subsection{偏差修正}
由于$v_0=0$所以数据初始的时候会与正常的数据小很多

对此,有:

对于第$t$天,我们不直接使用$v_t$作为数据,我们
使用$\frac{v_t}{1-\beta^t}$来作为每天的加权平均值

但是这个对于机器学习是非必要的,因为这个仅仅影响了前期的数据
,后面的数据和修正后的数据偏差不大。

\section{Momentum梯度下降法}
Momentum梯度下降法总是快于标准的梯度下降算法

设当前为第$t$次迭代,假设我们已经计算出来了$dw,db$正常的梯度下降法应该是:
$$w=w-\alpha dw$$
$$b=b-\alpha db$$
在Momentum梯度下降法中,我们使用一下的公式:
$$v_{dw}=\beta v_{dw}+(1-\beta)dw$$
$$v_{db}=\beta v_{db}+(1-\beta)db$$
$$w=w-\alpha v_{dw}$$
$$b=b-\alpha v_{db}$$
这是因为,每次的迭代一般都使得参数$W$更偏向与目标值,但是也会朝梯度下降垂直的方向上移动
,由于这些移动总是来回的摆动,所以可以使用均值的方法吧这些摆动消除掉,同时也可以增加梯度下降方向上的速度

这个算法中包含了两个超参数:$\alpha$与$\beta$
,其中$\beta$的常用值是$0.9$

这里的加权平均可以不用进行偏参修正

有的人使用的公式可能是不一样的:
$$v_{dw}=\beta v_{dw}+dw$$
$$v_{db}=\beta v_{db}+db$$
这里没有$(1-\beta)$,这样的话,在更新$w$和$b$的时候,$v_{dw}$与$v_{db}$就需要乘于一个$\frac1{1-\beta}$


\section{RMSprop算法}
全名为:"root mean square prop"

可以加速梯度下降

设当前为第$t$次迭代,有:
$$S_{dw}=\beta S_{dw}+(1-\beta)dw^2$$
$$S_{db}=\beta S_{db}+(1-\beta)db^2$$
$$w=w-\alpha\frac{dw}{\sqrt{S_{dw}}}$$
$$b=b-\alpha\frac{db}{\sqrt{S_{db}}}$$

在实际操作过程中,为了避免分母为零或者非常趋近与零的情况,我们在分母上加上一个很小的数$\varepsilon$
,例如$\varepsilon = 10^{-8}$

这个算法是起到一个归一化的作用

\section{Adam算法}
这个算法是上述两个算法的结合

同样对于第$t$次迭代,有:
$$v_{dw}=\beta_1 v_{dw}+(1-\beta_1)dw$$
$$v_{db}=\beta_1 v_{db}+(1-\beta_1)db$$
$$S_{dw}=\beta_2 S_{dw}+(1-\beta_2)dw^2$$
$$S_{db}=\beta_2 S_{db}+(1-\beta_2)db^2$$
在进行Adam算法的时候,一般要进行 偏差修正 
$$v_{dw}^{corrected}=\frac{v_{dw}}{1-\beta_1^t}$$
$$v_{db}^{corrected}=\frac{v_{db}}{1-\beta_1^t}$$
$$S_{dw}^{corrected}=\frac{S_{dw}}{1-\beta_2^t}$$
$$S_{db}^{corrected}=\frac{S_{db}}{1-\beta_2^t}$$
$$w=w-\alpha\frac{v_{dw}^{corrected}}{\sqrt{S_{dw}^{corrected}}+\varepsilon}$$
$$b=b-\alpha\frac{v_{db}^{corrected}}{\sqrt{S_{db}^{corrected}}+\varepsilon}$$
超参数$\alpha$通常要调试,可以取一系列的值进行测试
,$\beta_1$的缺省值通常是0.9, $\beta_2$通常推荐使用0.999,$\varepsilon = 10^{-8}$

\section{逐步减小学习率}
随着迭代的次数增加,学习率也逐步减小
,这样做可以在初始时有较大的步骤,但是在后期可以以很小的步骤去逼近目标值

\section{局部最优问题}
在三维空间中,使用梯度下降法,最可能会到达局部最优点,即一个极小值点

但是在高维的空间中,局部最优点往往不是那么容易到达,最容易到达的是鞍点

所以我们可以不用考虑局部最优的问题,转而考虑鞍点

但是由于mini-batch算法本身的特性,使得网络很容易就可以走出鞍点

我们应该重点考虑的应该是一些比较平缓的函数,例如高原一样

\chapter{超参数调试、Batch正则化和程序框架}

\section{调试处理}
如何系统地组织超参数调试过程的技巧

如何处理超参数

$\alpha$(学习率)\\$\beta$(momentum算法)\\
$\beta_1,\beta_2,\epsilon$(Adam算法)\\

还有 网络层数,每层节点个数,学习率递减参数

Mini-batch的大小

学习速率是最重要的\\其次是momentum算法中的参数
、网络节点个数、Mini-batch大小\\然后是网络层数,递减参数,最后才是Adam算法的参数

假如有两个超参数,可以随机在平面中选取25个点,然后代入

如果发现一个范围内的点效果更好,那么就可以集中在这个区域去取值

随机取值实在一个合适的范围内取值

在一个范围内可以均匀取值,也可以以其他的方式取值

例如,学习率如果范围为$[0.0001,1]$之间,应该在对数轴上取值

如: \\r = -4*np.random.rand()
\\$\alpha=10^r$

但是对于$\beta$,这是一个加权平均值的参数
,如果假设$\beta \in [0.9,0.999]$的话

我们可以让$1-\beta$在对数轴上取值

熊猫方法 和 鱼子酱方法

有你的计算机配置决定

\section{Batch归一化}

使得神经网络对超参数的选择更加的稳定

使得更容易训练网络和深层网络

在之前logistic回归的过程中,使用归一化处理使得训练集达到去平均的目的
$$\mu = \frac1m\sum x^{(i)}$$
$$x = x-\mu$$
$$\sigma^2 = \frac1m\sum x^{(i)2}$$
$$x = x/\sigma^2$$

问题:在深度神经网络中,是否可以归一化每一层的$a$
,从而更快的训练$w,b$

\subsection{单层网络中使用归一化}
通常的操作是归一化$z$,即在激活函数之间就进行归一化处理

设$z^{(1)}\dots z^{(m)}$是隐藏单元值,即$z^{[l](i)},i=1,2,\dots,m$,下面是处理单层的归一化操作:

计算平均值:
$$\mu = \frac1m\sum_iz^{(i)}$$
计算方差:
$$\sigma^2 = \frac1m\sum_i(z^{(i)}-\mu)^2$$
归一化:
$$z_{norm}^{(i)}=\frac{z^{(i)}-\mu}{\sqrt{\sigma^2+\epsilon}}$$

为了不使该层的单元是归一化的
$$\tilde z^{(i)} = \gamma z^{(i)}_{norm}+\beta$$
这里的$\gamma$和$\beta$是学习参数,这两个参数可以使我们可以随意的设置平均值和方差

如果我们想要把方差设置为$\sigma_1$,平均值为$\mu_1$那么,我们可以使:
$$\gamma = \sqrt{\sigma_1^2+\epsilon},\beta = \mu_1$$

这样的操作后,如果再使用sigma函数,可以让数据更好的散布在非线性的范围内。

\subsection{深度网络中使用batch归一化}

batchnorm简称为BN

$$x \xrightarrow{w^{[1]},b^{[1]}}z^{[1]}\xrightarrow[\text{BN}]{\beta^{[1]},\gamma^{[1]}}\tilde z\xrightarrow a^{[1]}=g^{[1]}(\tilde z^{[1]})\xrightarrow{w^{[2]},b{[2]}}z^{2}
\xrightarrow[BN]{\beta^{[2]},\gamma^{[2]}}\tilde z^{[2]}\to a^{[2]}\to \dots$$

参数:
$$w^{[1]},b^{[1]},w^{[2]},b^{[2]},\dots,w^{[L]},b^{[L]}$$
$$\beta^{[1]},\gamma^{[1]},\beta^{[2]},\gamma^{[2]},\dots,\beta^{[L]},\gamma^{[L]}$$
这里的$\beta$和Momentum,RMSprop算法,Adam算法的$\beta$不一样

这里的$\beta$也需要更新,即:
$$\beta^[l] = \beta^[l]-\alpha d\beta^[l]$$

这里的batch归一化通常是和mini-batch一起进行的,所以有:
$$X^{\{1\}}\xrightarrow{w^{[1]},b^{[1]}}z^{[1]}\xrightarrow[BN]{\beta^{[1]},\gamma^{[1]}}\tilde z^{[1]}\to g^{[1]}(\tilde z^{[1]})=a^{[1]}\xrightarrow{w^{[2]},b^{[2]}}z^{[2]}\to \dots$$
$$X^{\{2\}}\xrightarrow{w^{[1]},b^{[1]}}z^{[1]}\xrightarrow[BN]{\beta^{[1]},\gamma^{[1]}}\tilde z^{[1]}\to g^{[1]}(\tilde z^{[1]})=a^{[1]}\xrightarrow{w^{[2]},b^{[2]}}z^{[2]}\to \dots$$
$$\vdots$$

其实,当你使用batch的时候,$b$参数可以去掉,以为无论你的$b$是多少,最后都会被归一化然后在重新通过$\beta,\gamma$来去归一化。

所以有:
$$z^{[l]}=w^{[l]}a^{l-1}$$

$\beta^{[l]}$和$\gamma^{[l]}$的维度都是$(n^{[l]},1)$

步骤:
\begin{enumerate}
    \item 正向遍历每一个mini-batch
    \item 在遍历过程中使用batch归一化替代$z^{[l]}$为$\tilde z^{[l]}$
    \item 反向计算所有的$dw^{[l]},db^{[l]},d\beta^{[l]},d\gamma^{[l]}$
    \item 更新参数
\end{enumerate}
\subsection{Batch归一化原理}
通过Batch归一化可以加速学习,

可以使权重比网络更滞后或者更深层,

covariate shift

降低了隐藏单元变化的数量

batch归一化可以使$z^{[i]}$的均值和方差均保持不变,通过这样限制了前面的层会影响数值分布的程度,使得这些值更加的稳定

同时,batch归一化还有轻微的正则化的效果,因为它在每个隐藏层的激活值上增加了噪音
\section{测试时的mini-batch}
$$\mu = \frac1m\sum_iz^{(i)}$$
$$\sigma^2=\frac1m\sum_i(z^{(i)}-\mu)^2$$
$$z_{norm}^{(i)}=\frac{z^{(i)}-\mu}{\sqrt{\sigma^2+\epsilon}}$$
$$\tilde z^{(i)}=\gamma z^{(i)} + \beta$$

在mini-batch中,对于每一个mini-batch 有:
$$x^{\{1\}},x^{\{2\}},x^{\{3\}},\dots$$
$$\mu^{\{1\}[l]},\mu^{\{2\}[l]},\mu^{\{3\}[l]},\dots$$
此时我们使用加权平均值来计算,同时也可以在$\sigma$上使用

这样$\mu$和$\sigma$就相当于是在整个样本上计算的了

\section{Softmax回归}
可以识别多种类型的东西,在深度网络的最后一层,不再是单节点了

$$z^{[L]}=w^{[L]}a^{[L-1]}+b^{[L]}$$
$$t = e^{z^{[L]}}$$
$$a^{[L]}=\frac{e^{z^{[L]}}}{\sum_{i=1}^{n^{[L]}}t_i}$$
$$a^{[L]}_i=\frac{t_i}{\sum t_j}$$
$n^{[L]}$是最后一层的节点数量
$a^{[L]}$的每一个元素代表的是这一个类的概率

我们把从$z^{[L]}$到$a^{[L]}$的函数称为 Softmax激活函数
\subsection{Softmax训练器}
hard max:计算$z^{[L]}$的最大值

soft max:计算$a^{[L]}$的最大值

cost函数:
$$l(\hat y,y)=-\sum_{j=1}^{n^{[L]}}y_j\log\hat y_j$$

$$J(w^{[1]},b^{[1]},\dots)=\frac1m\sum_{i=1}^ml(\hat y^{(i)},y^{(i)})$$

这个过程和似然估计法类似

\subsection{Softmax反向传播}
$$dz^{[L]}=\hat y-y$$
\section{框架}

\begin{enumerate}
    \item Caffe/Caffe2
    \item CNTK
    \item DL4j
    \item Keras
    \item Lasagne
    \item mxnet
    \item PaddlePaddle
    \item TensorFlow
    \item Theano
    \item Torch
\end{enumerate}

\section{TensorFlow}
一个框架

如果成本函数为:
$$J=w^2-10w+25$$

\begin{lstlisting}[language = python]
    import numpy as np
    import tensorflow as tf
    w = tf.Variable(0,drype=tf.float32)#定义参数,并初始化
    cost = tf.add(tf.add(w**2,tf.multiply(-10.,w)),,25)#设置成本函数
    #cost = w**2-10*w+25#也可以
    train=tf.train.GradientDescentOptimizer(0.01).minimize(cost)#设置学习率,使用梯度下降法

    init = tf.global_variables_initializer()
    session = tf.Session()
    session.run(init)
    print(session.run(w))#输出w
    session.run(train)#做一步梯度下降法
    print(session.run(w))#输出w
    for i in range(1000):
        session.run(train)
    print(session.run(w))
\end{lstlisting}

训练集:
\begin{lstlisting}[language = python]
    x = tf.placeholder(tf.float32,[3,1])#表示后面会给x提供值
    coefficient = np.array([[1.],[-10.],[25.]])
    session.run(train,feed_dict={x:coefficients})
\end{lstlisting}
\end{document}